% see https://en.wikibooks.org/wiki/LaTeX/Glossary
% and https://www.overleaf.com/learn/latex/Glossaries

\newglossaryentry{inference rule}{
  name={inference rule},
  plural={inference rules},
  description={operate on expressions to produce new expressions}
}


\newglossaryentry{feed}{
  name={feed},
  plural={feeds},
  description={used as input for an inference rule. An inference rule may require one or more "feed" values. A feed does not contain a relation like equality or inequality}
}


\newglossaryentry{expression}{
  name={expression},
  plural={expressions},
  description={mathematical statements composed of symbols and operators. Each expression has at least three components: a left-hand side (LHS), a relation operator, and a right-hand side (RHS)}
}

\newglossaryentry{operator}{
  name={operator},
  plural={operators},
  description={transforms applied to symbols}
}

\newglossaryentry{symbol}{
  name={symbol},
  plural={symbols},
  description={variables and constants}
}

\newglossaryentry{relation operator}{
  name={relation operator},
  plural={relation operators},
  description={separates the LHS and RHS of an \gls{expression}} %Examples include $=$, $\lt$, $\leq$, $\gt$, $\geq$} 
  % TODO: why can't those be used in glossary?
}


\newglossaryentry{derivation}{
  name={derivation},
  plural={derivations},
  description={comprised of \glspl{step}}
}

\newglossaryentry{step}{
  name={step},
  plural={steps},
  description={relates \glspl{expression} by an \gls{inference rule}}
}
