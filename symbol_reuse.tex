\section{Reuse of Symbols\label{sec:symbol_reuse}}

 Symbol re-use is common in science and is addressed in the PDG by the use
  of unique identifiers.
  For example, \(c\) is commonly used to represent the speed of light, but
  the same symbol is also used in the quadratic equation
\begin{equation}
a x^2+b x+c =0.
\label{eq:quadratic}
\end{equation}
In Eq.~\ref{eq:quadratic}, \(c\) does not refer to the speed of light.
This is distinguishable in the Physics Derivation Graph because the \(c\)
  referring to the speed of light has a different numeric identifier than
  the \(c\) used in Eq.~\ref{eq:quadratic}.


Each time an expression, symbol, or inference rule is used in a step in the
  \pdg{}, the unique identifier is referenced.
  The referencing of unique node identifiers is what enables construction of
  the graph.
  For example, if Eq.~\ref{eq:period_freq_eq_1} is used in two distinct
  derivations, the same expression is referenced.
  Similarly, when the symbol \(T\) is used in any derivation, the associated
  numeric identifier refers to period.
  The symbol \(T\) used in another context would have a different unique
  numeric identifier.
This referencing of unique expressions, symbols, and inference rules is
done using a numeric identifier (alphanumeric for inference rules).
This uniqueness of expressions, symbols, and inference rules is a distinguishing feature of the \pdg{}.
The \pdg{} is designed to show one instance of each
expression, but feeds and inference rules may have multiple instances in the graph.