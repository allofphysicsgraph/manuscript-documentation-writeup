\section{Derivations that Span Physics\label{sec:span_physics}}

Spanning the diversity of applications in Physics  could refer to spanning different regimes -- fast and slow, big and small. Or the spanning could be of different topics -- classical mechanics, quantum, relativity. 
Those are different ways of describing the same approximations: 
\begin{itemize}
    \item The speed of an object being much less the speed of light ($v<<c$) is the classical approximation of special relativity.
    \item When the interactions of many particles are account for ($n>>1$), an ensemble gives rise to emergent behavior -- classical thermodynamics arises from quantum mechanics.
    \item Newtonian assumption of flat spacetime is an approximation of curved spacetime in general relativity.
    \item The amount of energy is assumed to be continuous in classical systems, though energy is quantized.
\end{itemize}
Another distinguishing feature of different topics in Physics is the use of assumptions:
\begin{itemize}
    \item assume a noiseless environment (for quantum mechanics).
    \item assume the mass is small enough that it does not deform space.
    \item assume a point particle (rather than a field).
    \item quantum mechanics: particles or waves
    \item Newtonian assumption that speed is unbounded.
    \item Newtonian assumption of flat space (rather than curved)
\end{itemize}

To interconnect the topics of Physics, derivations that bridge assumptions and approximations are the primary focus. Derivations within a topic, while supported by the \pdg{}, are lower priority.